% replace all text with your own text.
% in this template few examples are mention
\chapter{Methodology}
\label{ch:method} % Label for method chapter
Text Here

\section{Problem Description and Requirements}

The primary problem that the Tank Tactics project aims to address is the development of an online multiplayer tank battle game that overcomes the limitations of traditional multiplayer games, such as the need for manual port forwarding and IP address sharing. This task presents several challenges, including the need for seamless multiplayer connectivity, the creation of engaging gameplay mechanics, and the incorporation of advanced features to enhance the overall gaming experience.
\\
\noindent
\\
Developing an online multiplayer game like Tank Tactics requires a fast and efficient networking solution. The game must support multiple players interacting in real-time, which necessitates a secure, efficient, and low-level latency network. Any delay or lag in the game can significantly impact the user experience, making seamless networking crucial for the success of the project.
\\
\noindent
\\
Another significant challenge is the creation of engaging gameplay mechanics. Tank Tactics should feature intuitive tank controls, precise shooting mechanics, and a dynamic coin collection system. These core gameplay elements must be designed to provide an enjoyable and engaging experience for players.
\\
\noindent
\\
The incorporation of advanced features is also a key requirement for Tank Tactics. The game should include elements such as a real-time leaderboard, mini-map, healing zones, and bounty coins to add depth and variety to the gameplay. The specific context of the project also plays a vital role. Tank Tactics is designed to cater to fans of .io-style games and classic tank-based titles. The game should draw inspiration from these genres to create a familiar and intuitive experience for players. The target platform for the game is the Unity game engine, which influences the development process and the choice of networking technologies.
\\
\noindent
\\
Moving on to the requirements of the project, the functional requirements include seamless multiplayer connectivity, engaging gameplay mechanics, and the incorporation of advanced features. Seamless multiplayer connectivity is essential for allowing up to 20 players to interact in the same game environment without the need for manual port forwarding or IP address sharing. Engaging gameplay mechanics, such as intuitive tank controls and precise shooting are necessary and advanced features, like leaderboards, and mini-maps enhance the overall gaming experience.
\\
\noindent
\\
The non-functional requirements of the project include performance optimisation accessibility, and scalability. Performance optimisation is crucial for maintaining a smooth and responsive gaming experience for all players, even as the number of concurrent users increases. The game should also be accessible, with a user-friendly interface and minimal technical barriers to entry.
\\
\noindent
\\
In summary, the development of Tank Tactics presents several challenges and requirements that need to be addressed. The aim is to create an engaging, competitive, and accessible online multiplayer tank battle game by leveraging the power of Unity's multiplayer technologies and carefully considering these factors. 

\section{Technologies}
This section explores deeper into the key technologies employed in the development of Tank Tactics, providing a comprehensive overview of the Unity game engine, Visual Studio, C\# programming language, Unity Netcode, and Unity Gaming Services. By leveraging these powerful tools and frameworks, the project aims to create a seamless, engaging, and accessible online multiplayer gaming experience that effectively addresses the challenges and requirements outlined in section 3.1.

\subsection{Unity Game Engine}

Unity, as discussed in Chapter 2 (section 2.4.1), is a widely adopted cross-platform game engine that offers an extensive set of tools and features for game development. Its user-friendliness, versatility, and support for both 2D and 3D games across multiple platforms make it an ideal choice for the Tank Tactics project.
\\
\noindent
\\
One of the key advantages of using Unity for Tank Tactics is its component-based architecture and modular approach to game object composition. This design philosophy simplifies the development process and promotes code reusability by allowing developers to create and modify individual components without affecting the entire game object. This modular approach enables the Tank Tactics project to be efficiently created and iterate on gameplay mechanics, visual assets, and multiplayer functionality, streamlining the development workflow and reducing the time required to implement new features or make changes to existing ones.
\\
\noindent
\\
The Unity editor's intuitive interface and customisable layout further enhance the development experience. The editor provides a wide range of built-in tools for scene editing, asset management, and debugging, allowing developers to work more efficiently and effectively. For example, the scene view enables developers to visually design and manipulate game objects, while the hierarchy window provides a clear overview of the game object structure. The inspector window allows developers to modify the properties and components of selected game objects, and the project window serves as a central hub for managing and organizing game assets.
\\
\noindent
\\
In addition to these core features, Unity offers a vast ecosystem of plugins, extensions, and asset packages through the Unity Asset Store. These resources can significantly accelerate the development process by providing pre-built solutions, art assets, and tools that can be easily integrated into the project. Tank Tactics project can leverage these assets to enhance the game's visual quality, implement complex gameplay mechanics, or extend the functionality of the Unity editor to suit the project's specific needs.
\\
\noindent
\\
By leveraging the power and flexibility of the Unity game engine, the Tank Tactics project can benefit from a streamlined development process, a modular and reusable codebase, and a rich ecosystem of tools and assets. These advantages ultimately contribute to the creation of a polished, engaging, and accessible multiplayer gaming experience.

\subsection{Visual Studio and C\#}

Unity's primary scripting language is C\#, a powerful and versatile object-oriented programming language that is well-suited for game development. To write and debug C\# scripts efficiently, Visual Studio, a feature-rich integrated development environment (IDE), is the recommended choice for Unity projects.
\\
\noindent
\\
Visual Studio offers a wide range of productivity-enhancing features that streamline the coding process and help developers write clean, efficient, and maintainable code. One of the most notable features is IntelliSense, an intelligent code completion system that provides real-time suggestions and auto-completion for variables, methods, and classes as developers type. This feature significantly reduces the time required to write code and minimizes the risk of syntax errors, allowing developers to focus on implementing gameplay logic and mechanics.
\\
\noindent
\\
In addition to IntelliSense, Visual Studio provides real-time error detection and highlighting, which helps developers identify and resolve issues quickly. The IDE continuously analyzes the code as it is written, underlining potential errors and offering suggestions for fixes. This feature is particularly valuable in a complex project like Tank Tactics, where maintaining code quality and minimizing bugs is crucial for a smooth and enjoyable multiplayer experience.
\\
\noindent
\\
Visual Studio also offers powerful debugging tools that allow developers to investigate and resolve issues in their code. The debugger enables developers to pause the execution of the game at specific points, inspect variable values, and step through the code line by line to identify the source of problems. The IDE also provides advanced debugging features, such as conditional breakpoints and the ability to modify variable values during runtime, which can be invaluable when troubleshooting complex gameplay mechanics or multiplayer interactions.
\\
\noindent
\\
C\#'s object-oriented nature and extensive class library make it an ideal language for implementing intricate gameplay mechanics, and multiplayer network logic. The language supports essential object-oriented programming concepts, such as encapsulation, inheritance, and polymorphism, which allow developers to create modular, reusable, and maintainable code structures. By leveraging these features, the Tank Tactics project can be made with a codebase that is easy to understand, modify, and extend throughout the development process.
\\
\noindent
\\
Furthermore, C\# offers a wide range of built-in classes and libraries that simplify common programming tasks, such as string manipulation, file I/O, and collections. The language also provides support for advanced features like LINQ (Language Integrated Query), which enables developers to write expressive and concise queries for data manipulation and filtering. These features can significantly reduce the amount of boilerplate code required and allow developers to focus on implementing game-specific logic.
\\
\noindent
\\
By utilizing C\# and Visual Studio, the Tank Tactics project can benefit from a robust and efficient development environment that promotes code quality, maintainability, and extensibility. The combination of a powerful programming language and a feature-rich IDE empowers developers to create engaging gameplay mechanics, optimize performance, and ensure a smooth and enjoyable multiplayer experience for players.

\subsection{Unity Netcode and Unity Gaming Service}

Tank Tactics makes use of Unity's Netcode for GameObjects (NGO) framework and Unity Gaming Services (UGS) Relay \& Lobby to solve the multiplayer connectivity issues mentioned in Chapter 1 (section 1.2) and Chapter 2 (section 2.2).
\\
\noindent
\\
Unity Netcode for GameObjects (NGO) is a high-level networking library that abstracts the complexities of low-level network communication, allowing developers to focus on creating gameplay logic and synchronizing objects across the network. NGO is built on top of the Unity Transport Package, a low-level networking library that manages connections using UDP (User Datagram Protocol) and handles reliable and unreliable message sending.
\\
\noindent
\\
One of the key advantages of using NGO in Tank Tactics is its simplicity and ease of use. The framework provides a set of components and APIs that seamlessly integrate with Unity's existing GameObject and component system, making it easier for developers to add multiplayer functionality to their games. For example, the NetworkManager component acts as the core orchestrator for multiplayer sessions, handling tasks such as server startup, client connection, and scene management. The NetworkBehaviour component, on the other hand, enables developers to define which game objects and properties should be synchronized across the network.
\\
\noindent
\\
NGO also provides a range of built-in features that simplify common multiplayer tasks. These include object spawning, state synchronization, and remote procedure calls (RPCs). Object spawning allows the server to create game objects on all connected clients, ensuring that all players see the same objects in the game world. State synchronization keeps the game state consistent across all clients by automatically sending updates for synchronized variables whenever their values change. RPCs enable developers to define methods that can be invoked remotely by other clients or the server, facilitating communication and interaction between players.
\\
\noindent
\\
To further enhance the multiplayer experience and address the challenges associated with manual port forwarding and IP address sharing, Tank Tactics integrates Unity Gaming Services (UGS) Relay \& Lobby. UGS Relay is a managed service that provides a secure and scalable infrastructure for hosting multiplayer game sessions. By leveraging the Relay service, Tank Tactics can establish connections between players without requiring them to configure their routers or firewalls, eliminating the need for manual port forwarding and ensuring a seamless and accessible multiplayer experience.
\\
\noindent
\\
UGS Lobby, on the other hand, offers a convenient way for players to find and join multiplayer game sessions. The Lobby service allows developers to create and manage game lobbies programmatically, enabling players to browse available sessions, join existing lobbies, or create new ones. This feature streamlines the matchmaking process and provides a centralized hub for players to connect and interact with each other before starting a game.
\\
\noindent
\\
The combination of Unity Netcode for GameObjects and Unity Gaming Services Relay \& Lobby enables Tank Tactics to implement a client-hosted listen server model, as described in Chapter 2 (section 2.3.5). In this hybrid network topology, one of the players acts as both a client and a server, hosting the game for other players who connect as clients. This approach offers several benefits, such as reduced hosting costs and improved accessibility, while still maintaining the advantages of a client-server architecture, such as better security and simplified game state management. By utilising the power of Unity Netcode and Unity Gaming Services, Tank Tactics can deliver a smooth, responsive, and seamless multiplayer experience that effectively addresses the challenges associated with traditional multiplayer game development. These technologies align with the project's objectives, as outlined in Chapter 1 (section 1.3), and provide a robust foundation for implementing the game's core multiplayer features and mechanics.

\section{Implementations}

\subsection{Game Design}

\subsection{Project Setup}

\subsection{Multiplayer Architecture}

\subsection{Core Gameplay Mechanics}

\subsection{UGS Integration}
%\clearpage %  use command \clearpage when you want section or text to appear in the next page.


\section{Summary}
Write a summary of this chapter.

