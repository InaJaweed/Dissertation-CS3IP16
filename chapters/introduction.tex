\chapter{Introduction}
\label{ch:into} % This how you label a chapter and the key (e.g., ch:into) will be used to refer this chapter ``Introduction'' later in the report. 
% the key ``ch:into'' can be used with command \ref{ch:intor} to refere this Chapter.

%\textbf{Guidance on introduction chapter writing:} Introductions are written in the following parts:
%\begin{itemize}
 %   \item A brief  description of the investigated problem.
  %  \item A summary of the scope and context of the project, i.e., what is the background of the topic/problem/application/system/algorithm/experiment/research question/hypothesis/etc. under investigation/implementation/development [whichever is applicable to your project].
   % \item The aims and objectives of the project.
    %\item A description of the problem and the methodological approach adopted to solve the problem.
    %\item A summary of the most significant outcomes and their interpretations.
    %\item Organization of the report. 
%\end{itemize}


%Consult \textbf{your supervisor} to check the content of the introduction chapter. In this template, we only offer basic sections of an introduction chapter. It may  not be complete and comprehensive. Writing a report is a subjective matter, and a report's style and structure depend on the ``type of project'' as well as an individual's preference. This template suits the following project paradigms:
%\begin{enumerate}
 %   \item software engineering and software/web application development;
  %  \item algorithm implementation, analysis and/or application;  
   % \item science lab (experiment); and
    %\item pure theoretical development (not mention extensively).
%\end{enumerate}

%Use only a single \textbf{font} for the body text. We recommend using a clean and electronic document friendly font like \textbf{Arial} or \textbf{Calibri} for MS-word (If you create a report in MS word). If you use this template, DO NOT ALTER the template's default font ``amsfont default computer modern''. The default \LaTeX~font ``computer modern'' is also acceptable. 

%The recommended body text \textbf{font size} is minimum \textbf{11pt} and minimum one-half line spacing. The recommended figure/table caption font size is minimum 10pt. The footnote\footnote{Example footnote: footnotes are useful for adding external sources such as links as well as extra information on a topic or word or sentence. Use command \textbackslash footnote\{...\} next to a word to generate a footnote in \LaTeX.} font size is minimum 8pt. DO NOT ALTER the font setting of this template.   

%%%%%%%%%%%%%%%%%%%%%%%%%%%%%%%%%%%%%%%%%%%%%%%%%%%%%%%%%%%%%%%%%%%%%%%%%%%%%%%%%%%
\section{Background}
\label{sec:into_back}
Multiplayer online games have seen a significant rise in popularity and player engagement in recent years. However, in traditional multiplayer game development often faces challenges in such as then need for manual port-forwarding and IP address sharing, which can hinder accessibility and growth. Tank Tactics aims to address these limitations by leveraging Unity's Netcode for GameObjects (NGO) framework and Unity Gaming Services (UGS) Relay \& Lobby, which enables seamless multiplayer connectivity and self-hosting.
\subsection{Tank Tactics Game Overview}
The project focuses on the development of an online multiplayer game called Tank Tactics. The gameplay of Tank Tactics is inspired by popular .io style games such as Agar.io and Slither.io, as well as classic tank-based games like Tank Trouble and Wii Tanks. The game features a top-down shooter mechanic, where the player controls tanks and engages in combat.
\subsection{Unity Game Engine and Multiplayer Technologies}
Tank Tactics is developed using Unity, a widely used game engine platform that provides a wide range of tools and features for game development. To enable seamless multiplayer functionality, the project leverages two key technologies: Unity's Netcode for GameObjects (NGO) framework and Unity Gaming Services (UGS) Relay \& Lobby.
\subsubsection{Netcode for GameObjects (NGO)}
Netcode for GameObjects (NGO) is a high-level API provided by Unity that simplifies the process of building networked games. It abstracts away many of the low-level details of networking, allowing developers to focus on creating the gameplay logic and synchronising objects across the network, NGO handles tasks such as network communication, object spawning, and state synchronisation, making it easier to create multiplayer games.
\subsubsection{Unity Gaming Services (UGS) Relay \& Lobby}
Unity Gaming Services (UGS) Relay \& Lobby is a service provided by Unity that handles the hosting and matchmaking aspects of multiplayer games. UGS Relay provides a secure and scalable infrastructure for hosting multiplayer game sessions, eliminating the need for players to manually configure port-forward or share IP addresses. UGS Lobby, on the other hand, facilitates the matchmaking process allowing players to find and join available game sessions.
\subsection{Key Feature of Tank Tactics}
To keep the gameplay engaging and dynamic the game incorporates various features including:
\begin{itemize}
    \item Leaderboard: A real-time leaderboard that tracks top-performing players, creating a sense of competitiveness among players.
    \item Mini-map: A miniaturised map that provides players situational awareness, allowing them to monitor the map and the position of other players.
    \item Healing Zone: Designated areas on the map where players can replenish their tank's health.
    \item Collecting Coins: Players can collect coins scattered around the map to earn points on the leaderboard. Collected coins can be used as currency to shoot other players or access healing zones.
    \item Bounty Coins: These coins are dropped when players have collected 100 or more regular coins which players can collect.
\end{itemize}

%%%%%%%%%%%%%%%%%%%%%%%%%%%%%%%%%%%%%%%%%%%%%%%%%%%%%%%%%%%%%%%%%%%%%%%%%%%%%%%%%%%
\section{Problem statement}
\label{sec:intro_prob_art}
Traditional online multiplayer games often present technical challenges that can negatively impact the player experience and hinder the accessibility and growth of online games. These challenges include:
\begin{itemize}
    \item The requirement for manual port-forwarding: To enable incoming connections for the game, players frequently need to change their router settings to open particular ports manually. For many players, this can be time-consuming and technically difficult~\cite{network-architecting}.
    \item IP address sharing: Traditional online games frequently require sharing IP addresses to create direct connections between players. Players must provide their IP addresses to others, which presents privacy and security issues~\cite{issues-socialComputing}.
    \item Complex network configuration: Complex network settings may be necessary to set up and maintain a reliable multiplayer gaming environment. For those without extensive networking knowledge, navigating firewall settings, network address translation (NAT), and other technical issues may be challenging for players~\cite{develop-unity}.
\end{itemize}
These technical hurdles create barriers to entry for players, limiting the accessibility of online multiplayer games~\cite{port-forwarding}. As a result, many players might be prevented from participating in online gaming, which might hinder the growth and popularity of multiplayer games. The success and broad uptake of online games depend on resolving these issues and offering a smooth, intuitive, multiplayer gaming experience~\cite{multiplayer-networking}.
\\
By utilising NGO framework and UGS Relay \& Lobby, Tank Tactics seeks to address these issues and provide users with a more engaging and accessible multiplayer gaming experience. 

%%%%%%%%%%%%%%%%%%%%%%%%%%%%%%%%%%%%%%%%%%%%%%%%%%%%%%%%%%%%%%%%%%%%%%%%%%%%%%%%%%%
\section{Aims and objectives}
\label{sec:intro_aims_obj}

\textbf{Aims:} The primary aim of this project is to use NGO framework and UGS Relay \& Lobby to build Tank Tactics. The game aims to provide an enjoyable multiplayer experience for up to 20 players simultaneously, addressing the common challenges of traditional online games, such as the need for manual port forwarding and IP address sharing.\\
\\
\textbf{Objectives:}
\begin{enumerate}
    \item Implement Unity NGO framework and UGS Relay \& Lobby service to enable seamless self-hosting and real-time multiplayer connectivity, eliminating players' need to configure their networking settings or share IP addresses.
    \item Develop the core gameplay mechanics of Tank Tactics, including intuitive tank movements, accurate shooting mechanics, and a dynamic coin collection system.
    \item Incorporate advanced features that enhance the overall gameplay experience, such as:
    \begin{itemize}
        \item Leaderboard
        \item Min-map
        \item Healing Zones
        \item Bounty Coins
    \end{itemize}
    \item Ensure the game's overall design and mechanics are inspired by popular .io style games and classic tank-based titles, providing players an engaging and familiar gaming experience.
\end{enumerate}



%%%%%%%%%%%%%%%%%%%%%%%%%%%%%%%%%%%%%%%%%%%%%%%%%%%%%%%%%%%%%%%%%%%%%%%%%%%%%%%%%%%
\section{Solution approach}
\label{sec:intro_sol} % label of Org section
The Solution approach for developing Tank Tactics multiplayer game using Unity's Network for GameObject (NGO) framework and Unity Gaming Services (UGS) Relay \& Lobby. Involves the following steps:
\begin{enumerate}
    \item Implementing the NGO framework and UGS Relay \& Lobby service to enable seamless multiplayer connectivity and self-hosting without the need for manual port-forwarding or IP address sharing. This eliminates common technical barriers and provides an accessible multiplayer experience for players.
    \item Developing the core gameplay mechanics of Tank Tactics, including intuitive tank movements, accurate shooting mechanics, and a dynamic coin collection system. These mechanics will be designed to provide an engaging and enjoyable gameplay experience.
    \item Incorporating advanced features such as a real-time leaderboard, mini-map, healing zones, and bounty coins to enhance the overall gameplay experience and keep players engaged. These features add depth and variety to the game, encouraging players to strategise and compete.
    \item Ensuring the game's overall design and mechanics are inspired by popular .io style games and classic tank-based titles. This familiarity will help attract players and provide an intuitive gaming experience.
    \item Conducting thorough testing and optimization to ensure a smooth, stable, and enjoyable multiplayer experience for up to 20 players simultaneously.
\end{enumerate}

%%%%%%%%%%%%%%%%%%%%%%%%%%%%%%%%%%%%%%%%%%%%%%%%%%%%%%%%%%%%%%%%%%%%%%%%%%%%%%%%%%%
%\section{Summary of contributions and achievements} %  use this section 
%\label{sec:intro_sum_results} % label of summary of results
%Describe clearly what you have done/created/achieved and what the major results and their implications are. 


%%%%%%%%%%%%%%%%%%%%%%%%%%%%%%%%%%%%%%%%%%%%%%%%%%%%%%%%%%%%%%%%%%%%%%%%%%%%%%%%%%%
\section{Organization of the report} %  use this section
\label{sec:intro_org} % label of Org section
This dissertation is structured as follows: Chapter 2, Literature Review, explores the existing research and solutions related to multiplayer game development, networking frameworks, and game design principles relevant to Tank Tactics. Chapter 3, Methodology, describes the design and implementation process of Tank Tactics, including the use of the NGO framework, UGS Relay \& Lobby service, and the development of core gameplay mechanics and advanced features. Chapter 4, Results, presents the outcomes of the development process, showcasing the implemented features, multiplayer functionality, and overall gameplay experience. Chapter 5, Discussion and Analysis, evaluates the results, discussing the significance of the implemented features, the effectiveness of the solution approach, and any limitations encountered during the development process. Chapter 6, Conclusions and Future Work summarizes the key findings and contributions of the project and outlines potential areas for future development and improvement of Tank Tactics. Chapter 7, Reflection, offers a personal reflection on the learning experience gained throughout the project, discussing challenges faced, lessons learned, and the impact of the project on personal growth and future work in game development.

