\chapter{Introduction}
\label{ch:into} % This how you label a chapter and the key (e.g., ch:into) will be used to refer this chapter ``Introduction'' later in the report. 
% the key ``ch:into'' can be used with command \ref{ch:intor} to refere this Chapter.

%\textbf{Guidance on introduction chapter writing:} Introductions are written in the following parts:
%\begin{itemize}
 %   \item A brief  description of the investigated problem.
  %  \item A summary of the scope and context of the project, i.e., what is the background of the topic/problem/application/system/algorithm/experiment/research question/hypothesis/etc. under investigation/implementation/development [whichever is applicable to your project].
   % \item The aims and objectives of the project.
    %\item A description of the problem and the methodological approach adopted to solve the problem.
    %\item A summary of the most significant outcomes and their interpretations.
    %\item Organization of the report. 
%\end{itemize}


%Consult \textbf{your supervisor} to check the content of the introduction chapter. In this template, we only offer basic sections of an introduction chapter. It may  not be complete and comprehensive. Writing a report is a subjective matter, and a report's style and structure depend on the ``type of project'' as well as an individual's preference. This template suits the following project paradigms:
%\begin{enumerate}
 %   \item software engineering and software/web application development;
  %  \item algorithm implementation, analysis and/or application;  
   % \item science lab (experiment); and
    %\item pure theoretical development (not mention extensively).
%\end{enumerate}

%Use only a single \textbf{font} for the body text. We recommend using a clean and electronic document friendly font like \textbf{Arial} or \textbf{Calibri} for MS-word (If you create a report in MS word). If you use this template, DO NOT ALTER the template's default font ``amsfont default computer modern''. The default \LaTeX~font ``computer modern'' is also acceptable. 

%The recommended body text \textbf{font size} is minimum \textbf{11pt} and minimum one-half line spacing. The recommended figure/table caption font size is minimum 10pt. The footnote\footnote{Example footnote: footnotes are useful for adding external sources such as links as well as extra information on a topic or word or sentence. Use command \textbackslash footnote\{...\} next to a word to generate a footnote in \LaTeX.} font size is minimum 8pt. DO NOT ALTER the font setting of this template.   

%%%%%%%%%%%%%%%%%%%%%%%%%%%%%%%%%%%%%%%%%%%%%%%%%%%%%%%%%%%%%%%%%%%%%%%%%%%%%%%%%%%
\section{Background}
\label{sec:into_back}
The project focuses on the development of an online multiplayer game called Tank Tactics. Utilising Unity's Netcode for GameObjects (NGO) framework and Unity Gaming Services (UGS) Relay \& Lobby, which allows self-hosting without the need for port forwarding and sharing IP addresses. The gameplay of Tank Tactics is inspired by popular .io style games such as Agar.io and Slither.io, as well as classic tank-based games like Tank Trouble and Wii Tanks. The game features a top-down shooter mechanic, where the player controls tanks and engages in combat. To keep the gameplay engaging and dynamic the game incorporates various features including:
\begin{itemize}
    \item Leaderboard: A real-time leaderboard that tracks top-performing players, creating a sense of competitiveness among players.
    \item Mini-map: A miniaturised map that provides players situational awareness, allowing them to monitor the map and the position of other players.
    \item Healing Zone: Designated areas on the map where players can replenish their tank's health.
    \item Collecting Coins: Players can collect coins scattered around the map to earn points on the leaderboard. Collected coins can be used as currency to shoot other players or access healing zones.
    \item Bounty Coins: These coins are dropped when players have collected 100 or more regular coins which players can collect.
\end{itemize}

%%%%%%%%%%%%%%%%%%%%%%%%%%%%%%%%%%%%%%%%%%%%%%%%%%%%%%%%%%%%%%%%%%%%%%%%%%%%%%%%%%%
\section{Problem statement}
\label{sec:intro_prob_art}
Traditional online multiplayer games often present technical challenges that can negatively impact the player experience and hinder the accessibility and growth of online games. These challenges include:
\begin{itemize}
    \item The requirement for manual port-forwarding: To enable incoming connections for the game, players frequently need to change their router settings to open particular ports manually. For many players, this can be time-consuming and technically difficult~\cite{network-architecting}.
    \item IP address sharing: Traditional online games frequently require sharing IP addresses to create direct connections between players. Players must provide their IP addresses to others, which presents privacy and security issues~\cite{issues-socialComputing}.
    \item Complex network configuration: Complex network settings may be necessary to set up and maintain a reliable multiplayer gaming environment. For those without extensive networking knowledge, navigating firewall settings, network address translation (NAT), and other technical issues may be challenging for players~\cite{develop-unity}.
\end{itemize}
These technical hurdles create barriers to entry for players, limiting the accessibility of online multiplayer games~\cite{port-forwarding}. As a result, many players might be prevented from participating in online gaming, which might hinder the growth and popularity of multiplayer games. The success and broad uptake of online games depend on resolving these issues and offering a smooth, intuitive, multiplayer gaming experience~\cite{multiplayer-networking}.
\\
By utilising NGO framework and UGS Relay \& Lobby, Tank Tactics seeks to address these issues and provide users with a more engaging and accessible multiplayer gaming experience. 

%%%%%%%%%%%%%%%%%%%%%%%%%%%%%%%%%%%%%%%%%%%%%%%%%%%%%%%%%%%%%%%%%%%%%%%%%%%%%%%%%%%
\section{Aims and objectives}
\label{sec:intro_aims_obj}

\textbf{Aims:} The primary aim of this project is to use NGO framework and UGS Relay \& Lobby to build Tank Tactics. The game aims to provide an enjoyable multiplayer experience for up to 20 players simultaneously, addressing the common challenges of traditional online games, such as the need for manual port forwarding and IP address sharing.\\
\\
\textbf{Objectives:}
\begin{enumerate}
    \item Implement Unity NGO framework and UGS Relay \& Lobby service to enable seamless self-hosting and real-time multiplayer connectivity, eliminating players' need to configure their networking settings or share IP addresses.
    \item Develop the core gameplay mechanics of Tank Tactics, including intuitive tank movements, accurate shooting mechanics, and a dynamic coin collection system.
    \item Incorporate advanced features that enhance the overall gameplay experience, such as:
    \begin{itemize}
        \item Leaderboard
        \item Min-map
        \item Healing Zones
        \item Bounty Coins
    \end{itemize}
    \item Ensure the game's overall design and mechanics are inspired by popular .io style games and classic tank-based titles, providing players an engaging and familiar gaming experience.
\end{enumerate}



%%%%%%%%%%%%%%%%%%%%%%%%%%%%%%%%%%%%%%%%%%%%%%%%%%%%%%%%%%%%%%%%%%%%%%%%%%%%%%%%%%%
\section{Solution approach}
\label{sec:intro_sol} % label of Org section
Briefly describe the solution approach and the methodology applied in solving the set aims and objectives.

Depending on the project, you may like to alter the ``heading'' of this section. Check with you supervisor. Also, check what subsection or any other section that can be added in or removed from this template.

\subsection{A subsection 1}
\label{sec:intro_some_sub1}
You may or may not need subsections here. Depending on your project's needs, add two or more subsection(s). A section takes at least two subsections. 

\subsection{A subsection 2}
\label{sec:intro_some_sub2}
Depending on your project's needs, add more section(s) and subsection(s).

\subsubsection{A subsection 1 of a subsection}
\label{sec:intro_some_subsub1}
The command \textbackslash subsubsection\{\} creates a paragraph heading in \LaTeX.

\subsubsection{A subsection 2 of a subsection}
\label{sec:intro_some_subsub2}
Write your text here...

%%%%%%%%%%%%%%%%%%%%%%%%%%%%%%%%%%%%%%%%%%%%%%%%%%%%%%%%%%%%%%%%%%%%%%%%%%%%%%%%%%%
\section{Summary of contributions and achievements} %  use this section 
\label{sec:intro_sum_results} % label of summary of results
Describe clearly what you have done/created/achieved and what the major results and their implications are. 


%%%%%%%%%%%%%%%%%%%%%%%%%%%%%%%%%%%%%%%%%%%%%%%%%%%%%%%%%%%%%%%%%%%%%%%%%%%%%%%%%%%
\section{Organization of the report} %  use this section
\label{sec:intro_org} % label of Org section
Describe the outline of the rest of the report here. Let the reader know what to expect ahead in the report. Describe how you have organized your report. 

\textbf{Example: how to refer a chapter, section, subsection}. This report is organised into seven chapters. Chapter~\ref{ch:lit_rev} details the literature review of this project. In Section~\ref{ch:method}...  % and so on.

\textbf{Note:}  Take care of the word like ``Chapter,'' ``Section,'' ``Figure'' etc. before the \LaTeX command \textbackslash ref\{\}. Otherwise, a  sentence will be confusing. For example, In \ref{ch:lit_rev} literature review is described. In this sentence, the word ``Chapter'' is missing. Therefore, a reader would not know whether 2 is for a Chapter or a Section or a Figure.

