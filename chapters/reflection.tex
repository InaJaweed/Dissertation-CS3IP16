\chapter{Reflection}
\label{ch:reflection}
%%%%%%%%%%%%%%%%%%%%%%%%%%%%%%%
%% Please remove/replace text below
%%%%%%%%%%%%%%%%%%%%%%%%%%%%%%%
The development of Tank Tactics has been a critical learning experience, providing me with the opportunity to apply and expand my technical skills while also enhancing my problem-solving and decision-making abilities. Throughout the project, I developed a deeper understanding of game development, multiplayer networking, and engaging gameplay design.
\\
\noindent
\\
One of the most significant areas of growth was my proficiency with the Unity game engine and its multiplayer technologies. I learned how to leverage Unity's Netcode for GameObjects (NGO) framework and Unity Gaming Services (UGS) Relay \& Lobby to implement seamless online multiplayer functionality. Adopting and integrating these frameworks into the project allowed me to expand my knowledge of networking principles and real-time multiplayer interactions.
\\
\noindent
\\
Designing engaging gameplay mechanics and features for Tank Tactics was both challenging and rewarding. Creating intuitive tank controls, precise shooting mechanics, and a dynamic coin system required a deep understanding of game design principles and player experience. While developing these core mechanics was initially complex, the process of iterating and refining them to create an enjoyable experience was immensely satisfying.
\\
\noindent
\\
One of the major challenges I faced was the implementation of efficient networking solutions to ensure smooth and responsive multiplayer gameplay for up to 20 concurrent players. Ensuring real-time interactions without significant lag or delay was a complex task that required me to expand my knowledge of networking principles and optimization techniques. While I was able to implement a solution that met the project's requirements, I believe there is still room for improvement in this area, particularly in terms of scalability and performance for larger player counts.
\\
\noindent
\\
If I were to approach a similar problem in the future, I would allocate more time to the initial planning and design phase. A more detailed plan and thorough architecture design could have helped mitigate some of the challenges I encountered during the development process, especially with client-server communication and synchronization.
Reflecting on the project's aims and objectives outlined in Chapter 1, while Tank Tactics successfully met the core requirements of delivering an engaging online multiplayer tank battle game, there were some deviations from the initial plan. For instance, the complexity and scope of certain advanced features, such as leaderboards and mini-maps, were simplified over time as I gained a better understanding of the intricacies involved in their implementation.
\\
\noindent
\\
These changes, though not initially planned, ultimately contributed to the success of the game by allowing me to focus on delivering a polished and stable core experience. The lessons learned from navigating these challenges and making informed decisions will undoubtedly be valuable in my future game development endeavours. Overall, the Tank Tactics project had a significant impact on my learning experience and future work. It not only allowed me to apply and expand my technical skills in game development and multiplayer networking but also helped me develop crucial problem-solving, decision-making, and project management abilities. The knowledge and expertise gained from this project will serve as a solid foundation for my continued growth and success in the field of game development.