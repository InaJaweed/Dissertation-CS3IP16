\chapter{Discussion and Analysis}
\label{ch:evaluation}


\section{Summary}
This chapter provides an in-depth discussion and analysis of the results obtained from the development and implementation of Tank Tactics, the online multiplayer tank battle game. It evaluates the significance of the findings, highlights the key achievements, and examines the limitations encountered during the project. The chapter aims to enhance the reader's understanding of the project's outcomes and their implications within the context of the objectives outlined in Chapter 1.

\section{Significance of the Findings}
The successful implementation of Unity's Netcode for GameObjects (NGO) framework and Unity Gaming Services (UGS) Relay \& Lobby in Tank Tactics represents a significant achievement in addressing the challenges associated with traditional online multiplayer games. By eliminating the need for manual port forwarding and IP address sharing, as discussed in Chapter 2 (section 2.2.2), the game provides a seamless and accessible multiplayer experience for players.
\\
\noindent
\\
The engaging and dynamic gameplay mechanics, including intuitive tank controls, strategic coin collection, and intense combat, contribute to an immersive and rewarding experience for players. The incorporation of advanced features such as the real-time leaderboard, mini-map, healing zones, and bounty coins adds depth and variety to the gameplay, encouraging players to strategies and compete, as envisioned in Chapter 1 (section 1.1.3).
\\
\noindent
\\
The modular and well-structured approach to development, as described in Chapter 3 (section 3.3.2), ensured the maintainability and extensibility of the codebase. The separation of concerns through categorized scripts and prefabs facilitated easy debugging and future expansion of the game's features, contributing to the project's long-term sustainability.

\section{Evaluation of the Solution Approach}
The client-hosted listen server model employed in Tank Tactics, as discussed in Chapter 2 (section 2.3.5), proved to be an effective solution for the development of a smaller-scale multiplayer game. This hybrid approach combined aspects of client-server and peer-to-peer architectures, providing improved security and simplified game state management compared to a pure peer-to-peer topology, while minimizing the costs and complexities associated with dedicated server infrastructure.
\\
\noindent
\\
The integration of Unity's NGO and UGS streamlined the development process, allowing the project to focus on gameplay mechanics and player interactions while relying on Unity's robust networking infrastructure. This aligns with the project's objectives outlined in Chapter 1 (section 1.3) and demonstrates the effectiveness of leveraging modern technologies and frameworks to create an engaging and accessible online multiplayer gaming experience.

\section{Impact of Survey Results and Player Feedback}
The results of the two surveys conducted during the project's development, as presented in Chapter 4 (sections 4.4 and 4.5), provided valuable insights into the game's performance, player experience, and areas for improvement. The initial survey results highlighted the presence of bugs and a lack of enjoyment among players, serving as a crucial wake-up call for the development team.
\\
\noindent
\\
In response to these findings, significant efforts were made to refine the gameplay mechanics, squash bugs, and enhance the overall user experience. The positive outcomes of the post-improvement survey, with 100\% of participants finding the game fun and engaging, and a significant reduction in reported bugs, validated the effectiveness of these development efforts.
\\
\noindent
\\
The survey results played a pivotal role in guiding the project's direction and ensuring that the final product met the expectations and preferences of the target audience. By actively incorporating player feedback and iterating on the game's design and implementation, Tank Tactics was able to deliver an engaging and polished multiplayer experience.
\section{Summary}
This chapter discussed the significance of the findings obtained from the development and implementation of Tank Tactics, highlighting the key achievements and their implications within the context of the project's objectives. It evaluated the effectiveness of the solution approach, particularly the client-hosted listen server model and the integration of Unity's multiplayer technologies.
\\
\noindent
\\
Furthermore, the chapter analysed the impact of the survey results and player feedback, emphasising their crucial role in guiding the project's direction and ensuring the delivery of an engaging and polished multiplayer experience. By addressing these limitations and implementing the suggested improvements, Tank Tactics could evolve into a more robust, feature-rich, and engaging online multiplayer experience game.