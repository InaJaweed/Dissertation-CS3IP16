\chapter{Literature Review}
\label{ch:lit_rev} %Label of the chapter lit rev. The key ``ch:lit_rev'' can be used with command \ref{ch:lit_rev} to refer this Chapter.

Text Will GO HERE

\section{History of Gaming and Networking}
Section 2.1 provides a comprehensive overview of the history of gaming and networking, tracing the evolution of multiplayer gaming from its early beginnings to the present day. By examining the key milestones and technological advancements that have shaped the multiplayer gaming landscape, this section aims to contextualise the development of Tank Tactics. The subsection covers various aspects of multiplayer gaming history, including early multiplayer games, the role of mainframe computers and early computer networks, the impact of LAN-based and Internet-based gaming, and recent trends and challenges in the gaming industry. The section concludes by discussing how Tank Tactics builds upon this rich legacy while addressing contemporary challenges in multiplayer game development.

\subsection{Early Multiplayer Games}
The concept of multiplayer gaming dates back to the earliest days of electronic games. In 1958, William Higinbotham created Tennis for Two, which allowed two players to compete against each other using separate controllers connected to an oscilloscope \cite{arm2006networking}. This groundbreaking game showcased the potential for social interaction through electronic gaming. A few years later, in 1961, students at the Massachusetts Institute of Technology (MIT) developed Spacewar, a multiplayer space combat game that ran on a PDP-1 computer \cite{arm2006networking}. Spacewar introduced concepts such as player-versus-player combat and shared-screen multiplayer.

\subsection{Mainframe Computers and Early Computer Networks}
The 1970s and 1980s witnessed the emergence of online gaming communities through the use of mainframe computers and early computer networks, PLATO (Programming Logic for Automatic Teach Operations) was a notable example of a mainframe-based system that supported multiplayer gaming \cite{arm2006networking}. Developed at the Univeristy of Illinois, PLATO allowed users to log into the system remotely and interact through various applications, including multiplayer games like Empire and Airflight. These games followed a centralised architecture, with the mainframe handling the primary computation and communication while players connected using terminals with limited processing power \cite{arm2006networking}. PLATO's multiplayer games demonstrated the feasibility of remote gaming and laid the foundation for future online gaming platforms. 

\subsection{MultiUser Dungeons (MUD) and Arcade Multiplayer Gaming}
Text Here

\subsection{LAN-based Multiplayer Gaming}
The introduction of personal computers and the development of local area networks (LAN) technology in the 1990s revolutionised multiplayer gaming. LAN allowed multiple computers to be interconnected within a limited geographic area, enabling players to connect and play games. One of the most influential games of this era was Doom (1993), a first-person shooter (FPS) that introduced a networked multiplayer game \cite{arm2006networking}. Doom allowed up to four players to compete against each other over a LAN using the IPX protocol, which was commonly used in Novell networks \cite{Doom}. However, Doom's reliance on broadcast packets for communication led to network congestion issues, prompting the development of more efficient networking models, in subsequent games \cite{arm2006networking}. LAN parties, where players gathered with their computers to play multiplayer games together, became increasingly popular.

\subsection{Internet-based Multiplayer Gaming}
The release of Quake in 1996 marked a turning point in multiplayer gaming. Developed by id Software, Quake introduced a client-server architecture that allowed it to connect to dedicated game servers over the Internet \cite{arm2006networking}. This architecture addressed the limitation of peer-to-peer networking used in earlier games like Doom and enabled the creation of persistent online gaming communities \cite{Doom}. Quake's popularity led to the emergence of organised clans, competitive tournaments, and the rise of professional gaming \cite{arm2006networking}. The game's success also sparked the development of user-created modifications (mods), which expanded the game's content and gameplay possibilities \cite{Doom}.
Quake's impact on the gaming industry cannot be overstated, as it laid the groundwork for large-scale, Internet-based multiplayer games.

\subsection{Recent and Future of  Multiplayer Gaming}
The rapid advancements in technology have significantly improved the online gaming industry, reshaping the landscape and paving the way for innovative gaming experiences. The increasing accessibility and affordability of high-speed internet connection, coupled with the expansion of powerful gaming devices, have filed the growth of multiplayer gaming on a global scale.

\subsubsection{Mobile Gaming}
One of the most notable trends in recent years has been the rise of mobile gaming. The widespread adoption of smartphones and tablets has made gaming more accessible than ever before, allowing players to engage in multiplayer experiences on the go. The integration of social features and multiplayer capabilities in mobile games has fostered a sense of community and competitiveness among players, driving engagement and further fueling the market's growth.
\subsubsection{Virtual Reality (VR) and Augmented Reality (AR)}
Another key development in the multiplayer gaming space is the advancement of virtual reality (VR) technologies. VR and AR have the potential to revolutionize the way players interact with virtual works and each other. With the development of more powerful hardware and sophisticated software, VR headsets have become more accessible and user-friendly offering players a highly immersive and engaging experience \cite{advancement-onlinegaming}. The integration of haptic feedback and motion tracking technologies further enhances the sense of presence and interaction within these virtual environments, AR games such as Pokemon Go, have also gained significant popularity, blurring the lines between the real world and the virtual world by superimposing digital elements onto the player's surroundings \cite{advancement-onlinegaming}.

\subsubsection{Challanges}
However, the rapid growth of online gaming has also brought forth new challenges that need to be addressed. One of the most pressing concerns is data privacy and security. With millions of players engaging in online gaming platforms the amount of personal data being collected and stored has reached unprecedented levels \cite{future-onlinegaming}. Players are increasingly worried about the potential for their personal information to be compromised or exploited by cybercriminals. Game developed and publishers must prioritise data protection and implement robust security measures to safeguard player information and maintain trust within the gaming community.
\\
\noindent
\\
Another challenge facing the online gaming industry is the need for greater inclusivity and accessibility. Cross-platform compatibility has emerged as a crucial factor in promoting inclusivity, allowing players to connect and play with others regardless of the device they use \cite{future-onlinegaming}.

\subsection{Tank Tactics: Building Upon the Legacy of Multiplayer Gaming}
Tank Tactics builds upon the rich history of multiplayer gaming while incorporating modern technologies and design principles to create a unique and engaging experience. The game draws inspiration from early multiplayer titles such as Tennis for Two and Spacewar, which pioneered the concept of competitive gameplay between multiple players. However, Tank Tactics leverages the power of modern game engines, like Unity, and advanced networking frameworks such as NGO and UGS Relay \& Lobby, to provide a seamless and accessible multiplayer experience.
\\
\noindent
\\
In contrast to the early multiplayer games that were limited by the technology of their time, Tank Tactics takes advantage of the widespread availability of high-speed internet connections and the increase of gaming devices to reach global audience. The game's focus on real-time, fast-paced tank combat draws inspiration from classic titles like Tank Trouble and Wii Tanks while introducing modern features such as leaderboards, mini-map, and dynamic gameplay elements like healing zones and bounty coins.
\\
\noindent
\\
Furthermore, Tank Tactics differentiated itself from traditional multiplayer games by eliminating the need for manual port-forwarding and IP address sharing, which were common challenges in the early days of online gaming. By utilising Unity's NGO framework and UGS Relay \& Lobby, the game provides a seamless and user-friendly multiplayer experience, allowing players to easily connect and play.
\\
\noindent
\\
In summary, Tank Tactics represents a modern take on the multiplayer gaming legacy, combining classic gameplay elements with cutting-edge technologies and design principles to deliver an engaging accessible experience for players.

% PLEAE CHANGE THE TITLE of this section
\section{Online Multiplayer Game Development Challenges} 
% Note the use of \cite{} and \citep{}
  
Developing an online multiplayer game presents a unique set of challenges compared to single-player games. \cite{adding-mutliplayer} highlights that adding online multiplayer functionality to a game can double the amount of work required for programming. One of the primary challenges is connecting players over the internet, which involves establishing a connection between players, organising the network topology, and handling issues related to Network Address Translation (NAT) \cite{network-architecting}.
\\
\noindent
\\
Traditional online multiplayer games often require players to manually configure port forwarding and share IP addresses, which can be technically challenging and hinder accessibility \cite{network-architecting}. Additionally, complex network configurations may be necessary to set up and maintain a reliable multiplayer environment, which can be difficult for players without extensive networking knowledge \cite{develop-unity}.
\\
\noindent
\\
These technical hurdles create barriers to entry for players, limiting the accessibility of online multiplayer games \cite{port-forwarding}. As a result, many players might be prevented from participating in online gaming, The success and broad uptake of online games depend on resolving these issues and offering a smooth, intuitive multiplayer gaming experience \cite{multiplayer-networking}.
% PLEAE CHANGE THE TITLE of this section
\section{Network Topologies in Online Multiplayer Games }

Text Here

\section{Matchmaking and Lobby Systems}

Text Here

\section{Unity Game Engine and Multiplayer Technologies}

% A possible section of you chapter
\section{Critique of the review} % Use this section title or choose a betterone
Describe your main findings and evaluation of the literature. ~\\

% Pleae use this section
\section{Summary} 
Write a summary of this chapter~\\
