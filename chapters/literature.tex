\chapter{Literature Review}
\label{ch:lit_rev} %Label of the chapter lit rev. The key ``ch:lit_rev'' can be used with command \ref{ch:lit_rev} to refer this Chapter.

Text Will GO HERE

\section{History of Gaming and Networking}
Section 2.1 provides a comprehensive overview of the history of gaming and networking, tracing the evolution of multiplayer gaming from its early beginnings to the present day. By examining the key milestones and technological advancements that have shaped the multiplayer gaming landscape, this section aims to contextualise the development of Tank Tactics. The subsection in 2.1 covers various aspects of multiplayer gaming history, including early multiplayer games (2.1.1), the role of mainframe computers and early computer networks (2.1.2), the impact of MultiUser Dungeons (MUDs) and arcade multiplayer gaming (2.1.3), the impact of LAN-based (2.1.4) and Internet-based gaming (2.1.5), and recent trends and challenges in the gaming industry (2.1.6). The section, concludes by discussing how Tank Tactics builds upon this rich legacy while addressing contemporary challenges in multiplayer game development (2.1.7).

\subsection{Early Multiplayer Games}
The concept of multiplayer gaming dates back to the earliest days of electronic games. In 1958, William Higinbotham created Tennis for Two, which allowed two players to compete against each other using separate controllers connected to an oscilloscope \cite{arm2006networking}. This groundbreaking game showcased the potential for social interaction through electronic gaming. A few years later, in 1961, students at the Massachusetts Institute of Technology (MIT) developed Spacewar, a multiplayer space combat game that ran on a PDP-1 computer \cite{arm2006networking}. Spacewar introduced concepts such as player-versus-player combat and shared-screen multiplayer.

\subsection{Mainframe Computers and Early Computer Networks}
The 1970s and 1980s witnessed the emergence of online gaming communities through the use of mainframe computers and early computer networks, PLATO (Programming Logic for Automatic Teach Operations) was a notable example of a mainframe-based system that supported multiplayer gaming \cite{arm2006networking}. Developed at the Univeristy of Illinois, PLATO allowed users to log into the system remotely and interact through various applications, including multiplayer games like Empire and Airflight. These games followed a centralised architecture, with the mainframe handling the primary computation and communication while players connected using terminals with limited processing power \cite{arm2006networking}. PLATO's multiplayer games demonstrated the feasibility of remote gaming and laid the foundation for future online gaming platforms. 

\subsection{MultiUser Dungeons (MUDs) and Arcade Multiplayer Gaming}
The 1980s saw the rise of MultiUser Dungeons (MUDs), text-based virtual worlds where players could interact with the environment and each other through commands entered via a terminal \cite{bartle1990}. MUDs were inspired by the popularity of tabletop role-playing games like Dungeons \& Dragons and combined elements of storytelling, exploration, and social interaction \cite{Doom}. Games like MUD1, developed by Richard Bartle and Roy Trubshaw at the Univerity of Essex, allowed players to connect to a central server using Telnet, a protocol for remote terminal communication \cite{arm2006networking}. MUDs introduced concepts such as persistent worlds, player progression, and social interaction, which would later be expanded upon in graphical MMORPGs \cite{bartle2004designing}.
\\
\noindent
\\
MUDs utilised a client-server architecture, connecting players to the MUD server using a Telnet program. While the Telnet client itself was quite basic, the server-side game could be quite sophisticated, tracking the state of the virtual world, managing player's interactions and updating the game state in real time. The popularity of MUDs in academic circles helped spur innovations in game design, online communities, and server infrastructure.
\\
\noindent
\\
Concurrent with the development of MUDs, the golden age of arcade gaming in the 1970s and 1980s brought multiplayer gaming experience to the masses. Arcade machines, which were coin-operated and located in public spaces like shopping malls and amusement parks, offered a social gaming experience where players could compete against each other in the same physical location \cite{donovan2010replay}. Games like Pong (1972) and Space Invaders (1978) popularised the concept of head-to-head competition, while titles like Atari Football (1978) and Gauntlet(1985) introduced cooperative multiplayer gameplay \cite{arm2006networking}.
\\
\noindent
\\
While the technology underpinning arcade games was relatively simple compared to modern standards, the games showcased various innovative game mechanics and design elements. From pattern-based gameplay of Space Invaders to the scrolling levels of Super Mario Bros., arcade games laid the foundation for many of the game genres and conventions that are granted for today.
\\
\noindent
\\
The combination of MUDs and arcade games in the 1980s represented a significant step forward in the evolution of multiplayer gaming. MUDs showed the compelling multiplayer experience could be delivered over a network connection, while arcades demonstrated the social appeal and commercial viability of gaming as a shared activity.

\subsection{LAN-based Multiplayer Gaming}
The introduction of personal computers and the development of local area networks (LAN) technology in the 1990s revolutionised multiplayer gaming. LAN allowed multiple computers to be interconnected within a limited geographic area, enabling players to connect and play games. One of the most influential games of this era was Doom (1993), a first-person shooter (FPS) that introduced a networked multiplayer game \cite{arm2006networking}. Doom allowed up to four players to compete against each other over a LAN using the IPX protocol, which was commonly used in Novell networks \cite{Doom}. However, Doom's reliance on broadcast packets for communication led to network congestion issues, prompting the development of more efficient networking models, in subsequent games \cite{arm2006networking}. LAN parties, where players gathered with their computers to play multiplayer games together, became increasingly popular.

\subsection{Internet-based Multiplayer Gaming}
The release of Quake in 1996 marked a turning point in multiplayer gaming. Developed by id Software, Quake introduced a client-server architecture that allowed it to connect to dedicated game servers over the Internet \cite{arm2006networking}. This architecture addressed the limitation of peer-to-peer networking used in earlier games like Doom and enabled the creation of persistent online gaming communities \cite{Doom}. Quake's popularity led to the emergence of organised clans, competitive tournaments, and the rise of professional gaming \cite{arm2006networking}. The game's success also sparked the development of user-created modifications (mods), which expanded the game's content and gameplay possibilities \cite{Doom}.
Quake's impact on the gaming industry cannot be overstated, as it laid the groundwork for large-scale, Internet-based multiplayer games.

\subsection{Recent and Future of  Multiplayer Gaming}
The rapid advancements in technology have significantly improved the online gaming industry, reshaping the landscape and paving the way for innovative gaming experiences. The increasing accessibility and affordability of high-speed internet connection, coupled with the expansion of powerful gaming devices, have filed the growth of multiplayer gaming on a global scale.

\subsubsection{Mobile Gaming}
One of the most notable trends in recent years has been the rise of mobile gaming. The widespread adoption of smartphones and tablets has made gaming more accessible than ever before, allowing players to engage in multiplayer experiences on the go. The integration of social features and multiplayer capabilities in mobile games has fostered a sense of community and competitiveness among players, driving engagement and further fueling the market's growth.
\subsubsection{Virtual Reality (VR) and Augmented Reality (AR)}
Another key development in the multiplayer gaming space is the advancement of virtual reality (VR) technologies. VR and AR have the potential to revolutionize the way players interact with virtual works and each other. With the development of more powerful hardware and sophisticated software, VR headsets have become more accessible and user-friendly offering players a highly immersive and engaging experience \cite{advancement-onlinegaming}. The integration of haptic feedback and motion tracking technologies further enhances the sense of presence and interaction within these virtual environments, AR games such as Pokemon Go, have also gained significant popularity, blurring the lines between the real world and the virtual world by superimposing digital elements onto the player's surroundings \cite{advancement-onlinegaming}.

\subsubsection{Challenges}
However, the rapid growth of online gaming has also brought forth new challenges that need to be addressed. One of the most pressing concerns is data privacy and security. With millions of players engaging in online gaming platforms the amount of personal data being collected and stored has reached unprecedented levels \cite{future-onlinegaming}. Players are increasingly worried about the potential for their personal information to be compromised or exploited by cybercriminals. Game developed and publishers must prioritise data protection and implement robust security measures to safeguard player information and maintain trust within the gaming community.
\\
\noindent
\\
Another challenge facing the online gaming industry is the need for greater inclusivity and accessibility. Cross-platform compatibility has emerged as a crucial factor in promoting inclusivity, allowing players to connect and play with others regardless of the device they use \cite{future-onlinegaming}.

\subsection{Tank Tactics: Building Upon the Legacy of Multiplayer Gaming}
Tank Tactics builds upon the rich history of multiplayer gaming while incorporating modern technologies and design principles to create a unique and engaging experience. The game draws inspiration from early multiplayer titles such as Tennis for Two and Spacewar, which pioneered the concept of competitive gameplay between multiple players. However, Tank Tactics leverages the power of modern game engines, like Unity, and advanced networking frameworks such as NGO and UGS Relay \& Lobby, to provide a seamless and accessible multiplayer experience.
\\
\noindent
\\
In contrast to the early multiplayer games that were limited by the technology of their time, Tank Tactics takes advantage of the widespread availability of high-speed internet connections and the increase of gaming devices to reach global audience. The game's focus on real-time, fast-paced tank combat draws inspiration from classic titles like Tank Trouble and Wii Tanks while introducing modern features such as leaderboards, mini-map, and dynamic gameplay elements like healing zones and bounty coins.
\\
\noindent
\\
Furthermore, Tank Tactics differentiated itself from traditional multiplayer games by eliminating the need for manual port-forwarding and IP address sharing, which were common challenges in the early days of online gaming. By utilising Unity's NGO framework and UGS Relay \& Lobby, the game provides a seamless and user-friendly multiplayer experience, allowing players to easily connect and play.
\\
\noindent
\\
In summary, Tank Tactics represents a modern take on the multiplayer gaming legacy, combining classic gameplay elements with cutting-edge technologies and design principles to deliver an engaging accessible experience for players.

% PLEAE CHANGE THE TITLE of this section
\section{Online Multiplayer Game Development Challenges} 
% Note the use of \cite{} and \citep{}
Developing an online multiplayer game presents a unique set of challenges compared to single-player games. \cite{adding-mutliplayer} highlights that adding online multiplayer functionality to a game can double the amount of work required for programming. This section explores the various challenges developers face when creating online multiplayer games, including connecting players (2.2.1), manual port forwarding and IP sharing (2.2.2), complex network configurations (2.2.3) scalability and performance (2.2.4) and security (2.2.5). The section concludes by discussing how Tank Tactics addresses these challenges (2.2.6).

\subsection{Connecting Players}
Connecting players over the internet is a complex task that involves multiple layers of communication. At the most basic level, players' devices must establish a connection using a specific network protocol, such as Transmission Control Protocol (TCP) or User Datagram Protocol (UDP) \cite{network-architecting}. The choice of protocol depends on the game's requirement for speed, reliability, and error correction. Once a connection is established, the game must manage the flow of data between players, ensuring that updates are sent and received on time.
\\
\noindent
\\
Organising the network topology is another crucial aspect of connecting players. Developers must choose between a client-server architecture, where a central server manages the game state and communication between players, or a peer-to-peer architecture, where players communicate directly with each other \cite{develop-unity}. Each approach has its advantages and disadvantages in terms of scalability, performance, and security.
\\
\noindent
\\
Network Address Translation (NAT) is a common issue that complicates player connectivity. NAT is a technique used by routers to allow multiple devices on a local network to share a single public IP address. This can make it difficult for players to establish a direct connection, as incoming traffic must be correctly routed to the appropriate device \cite{port-forwarding}. Developers must implement techniques such as NAT traversal or relay servers to overcome this challenge.

\subsection{Manual Port Forwarding and IP Sharing}
Port forwarding is a technique used to allow incoming connection to a specific device on a local network. In the context of online gaming, players may need to manually configure their routers to forward incoming traffic on specific ports to their gaming devices \cite{network-architecting}. This process can be technically challenging, requiring the player to access their router's settings and correctly input the necessary information. Incorrect configuration can lead to connectivity issues or even expose the player's devices to security risks.
\\
\noindent
\\
IP sharing is another common requirement in traditional online multiplayer games. To establish a direct connection between players, each player must know the IP address of the other player \cite{multiplayer-networking}. This can raise privacy and security concerns, as the player must share their IP addresses with others, potentially exposing themselves to malicious actors. Developers can mitigate these risks by implementing secure communication channels and using techniques like encryption and authentication.

\subsection{Complex Network Configurations}
Setting up a reliable multiplayer gaming environment often requires players to configure their network settings, including firewall rules, port forwarding, and Quality of Service (QoS) settings \cite{develop-unity}. Firewalls are designed to block unauthorised incoming traffic, which can interfere with the game connections if not properly configured. Players may need to create exceptions in their firewall settings to allow the game to communicate with other players and servers.
\\
\noindent
\\
NAT can also complicate the setup process, as players may need to configure their routers to see specific NAT types (e.g. open, moderate, or strict) to ensure compatibility with the game's networking requirements \cite{port-forwarding}. Some games may require players to enable Universal Plug and Play (UPnP) or manually configure port forwarding to establish the connection.
\\
\noindent
\\
Developers can simplify the setup process by providing clear instructions, automatic configuration tools, or implementing techniques like NAT traversal and relay servers to minimise the need for manual configuration \cite{network-architecting}. However, the wide variety of network setups and hardware configurations makes it challenging to create a one-size-fits-all solution.

\subsection{Scalibility and Performance}
As online multiplayer grows in popularity, they must be able to handle an increasing number of concurrent players without sacrificing performance. Scalability refers to a game's ability to accommodate a growing player base while maintaining a high level of service \cite{multiplayer-networking}. This requires careful planning and optimisation of the game's architecture, including server infrastructure, network protocol, and game logic.
\\
\noindent
\\
Developers must choose the appropriate network topology (e.g. client-server or peer-to-peer) and design their server infrastructure to handle the expected player load \cite{network-architecting}. This may involve using techniques like load balancing, sharing, or cloud computing to distribute the workload across multiple servers, The game's network protocol must also be optimised to minimise latency and packet loss, ensuring a smooth and representative gaming experience.
\\
\noindent
\\
Performance optimisation is crucial for maintaining player satisfaction and retention. Developers must continuously monitor and profile the game's performance, identifying and addressing bottlenecks in the game logic, network communication, and rendering pipeline \cite{develop-unity}. This may involve techniques like data compression, interpolation, or client-side prediction to minimise the impact of network latency on gameplay. 

\subsection{Security}
Online multiplayer games must also protect against security threats, such as denial-of-service (DoS) attacks, which can overload servers and disrupt gameplay, and data breaches, which can expose player information and undermine trust in the game and its developers \cite{multiplayer-networking}. Developers must follow best practices for secure coding, regular patch vulnerabilities, and employ strong authentication and authorisation mechanisms to protect player accounts and data.
\\
\noindent
\\
Balancing security and performance is an ongoing challenge, as security measures can introduce overhead and latency that may impact gameplay. Developers must carefully consider the trade-offs and implement security solutions that provide necessary protection without compromising the player experience \cite{network-architecting}. 

\subsection{Tank Tactics: Addressing Multiplayer Game Development Challenges}
Tank Tactics addresses several of the challenges mentioned in the previous subsections by leveraging modern technologies and frameworks. By utilising the NGO framework and UGS Relay \& Lobby, the game eliminates the need for manual port-forwarding and IP address sharing. This significantly reduces the technical barriers for players and improves accessibility, as players no longer need to configure their routers or share personal information to establish connections.
\\
\noindent
\\
The use of UGS Relay \& Lobby also simplifies the process of connecting players, as it handles the complexities of NAT traversal and provides a secure, scalable infrastructure for hosting multiplayer game sessions. This allows players to easily find and join available game sessions without the need for complex network configurations.
\\
\noindent
\\
Furthermore, by leveraging the capabilities of the Unity game engine and its associated services, Tank Tactics can focus on optimising performance and scalability. The game can take advantage of Unity's built-in features for efficient network communication, object synchronisation, and game state management, reducing the development effort required to create a smooth and responsive multiplayer experience,
\\
\noindent
\\
In terms of security, Tank Tactics benefits from the security measures and best practices implemented by Unity and its services The use of secure communication protocols, encryption, and authentication mechanisms helps protect player data and prevent unauthorised access to game servers.

% PLEASE CHANGE THE TITLE of this section
\section{Network Topologies in Online Multiplayer Games }
The choice of network topology is a fundamental design decision when engineering a networked multiplayer game. The network topology describes how the computer and gaming devices are connected over the network \cite{multiplayer-networking}. The two main types of network topologies used in online multiplayer games are client-server and peer-to-peer.

\subsection{Client-Server Topology}
In a client-server topology, players connect to a central server that coordinates the game state and relays data between players. The server is the authoritative source of the game state. Clients send their input commands to the server, and the server validates these commands, simulates the game, and sends back the updated state to the clients \cite{network-architecting}.
\\
\noindent
\\
Advantages of client-server topology include:
\begin{itemize}
    \item Better security and cheat prevention since a trusted dedicated server has authority over the simulation
    \item Ability to handle a large number of players (scalability)
    \item Simplifed game state management and synchronisation
\end{itemize}
\noindent
\\
Disadvantages of client-server topology include: 
\begin{itemize}
    \item Cost of running dedicated servers
    \item Server being a single point of failure
    \item Potential for higher latency due to the extra hop between clients and server
\end{itemize}
\noindent
\subsubsection{Tribes Networking Model:}
Starsiege: Tribes, released in 1998, is an example of a game using a client-server model \cite{frohnmayer2000tribes}. In Tribes, a single central server maintains the authoritative game state and communicates with up to 128 clients. The ghost manager system on the server determines which objects are relevant to each client and only sends those. This relevant filtering was key to reducing bandwidth sage and scaling support to 128 concurrent players.

\subsection{Peer-to-Peer Topology}
In a peer-to-peer (P2P) topology, each player's computer connects directly to every other player's computer without a central server. Game state and simulation are distributed across all participating machines \cite{multiplayer-networking}.
\\
\noindent
\\
Advantages of P2P topology include:
\begin{itemize}
    \item Reduced infrastructure and hosting costs
    \item Potentially lower latency due to direct connections between peers
    \item No single point of failure
\end{itemize}
\noindent
\\
Disadvantages of P2P topology include:
\begin{itemize}
    \item Increased complexity in game state management and synchronisation
    \item Difficulty in preventing cheating and ensuring fair play
    \item Limited scalability due to exponential growth in the number of connections.
\end{itemize}
\noindent
\subsubsection{Deterministic Lockstep Model:}
Age of Empires, released in 1997, used a peer-to-peer topology with a deterministic lockstep networking model \cite{bettner20011500}. In Age of Empires, the game does not send the game state over the network - instead, it sends the player commands. The game simulation is deterministic, meaning that as long as every peer reviewers every command in the same order, the simulation will not diverge.
\\
\noindent
\\
Techniques used in the deterministic lockstep model include:
\begin{itemize}
    \item Command Queue: Each peer maintains a queue of commands to execute. Commands are only executed once all payers have acknowledged receiving them.
    \item Turn Timers: The game is divided into turns, with each turn lasting a fixed amount of time (e.g. 200ms). Peers send their commands for the turn to each other and then simulate the turn deterministically.
    \item Sync Checking: peers periodically compare their game state to ensure synchronisation and detect any divergence.
\end{itemize}
The deterministic lockstep model works well for games with a relatively small number of players and low update frequency, such as turn-based or slow-paced real-time strategy games. However, it can introduce latency and may not scale well to fast-paced games or a large number of players.

\subsection{Hybrid Topologies}
\cite{network-architecting} note that modern online games often use a combination or hybrid of client-server and peer-to-peer topologies to balance the advantages and disadvantages of each.
\noindent
\subsubsection{Listen Server:}
In a listen server mode, one player acts as the server, and the rest connects as clients to them. This avoids the infrastructure cost of dedicated servers while still providing some of the benefits of the client-server model, such as improved security and state management. However, the host player may have an unfair advantage in terms of latency, and the game is dependent on the host's connection quality.

\subsubsection{Broadcast Server:}
\cite{bettner20011500} discuss using a broadcast server to allow Age of Empires to scale to more than 4 players in a peer-to-peer topology. In this model, a server is used to broadcast game state updates to all peers, but the peers still simulate the game deterministically based on the received state updates. This reduces the number of connections each peer needs to maintain, improving scalability.

\subsection{Overall}
Client-server and peer-to-peer network topologies each have strengths and weaknesses. Client-server is the most common modern approach for online multiplayer games due to its ability to robustly scale to more players and provide strong security cheat prevention. However, peer-to-peer remains viable for smaller-scale games and those that wish to avoid the infrastructure costs of dedicated servers.
\\
\noindent
\\
Hybrid topologies that combine elements of client-server and peer-to-peer can offer a balance of their respective benefits. However, careful engineering is required with any topology to build a game that plays smoothly online, considering factors such as bandwidth usage, latency, state synchronisation, and cheat prevention.

\subsection{Tank Tactics: Client-Hosted Listen Server Model}
Tank Tactics employs a client-hosted listen server model, which is a type of hybrid topology that combines aspects of client-server and peer-to-peer architectures. In this model, one of the players acts as both a client and a server, hosting the game for other players who connect as clients.
\\
\noindent
\\
The listen server model used in Tank Tactics builds upon the traditional client-server topology by eliminating the need for dedicated server infrastructure. This reduces hosting costs and allows players to set up matches more easily. However, it still maintains many of the benefits of client-server architecture, such as improved security and simplified game state management compared to a pure peer-to-peer topology.
\\
\noindent
\\
Unlike the deterministic lockstep model used in Age of Empires, Tank Tactics does not rely on deterministic simulation across all clients. Instead, the listen server acts as the authoritative source of the game state, similar to a dedicated server in a client-server topology. This allows Tank Tactics to support faster-paced, real-time gameplay without the latency issues that can arise in a deterministic lockstep model.
\\
\noindent
\\
However, the client-hosted listen server model also has some disadvantages compared to a dedicated client-server architecture. The host player may have an advantage in terms of latency, as they do not need to spend updates to themselves. Additionally, the game's performance and stability are dependent on the host player's connection quality.
\\
\noindent
\\
Despite these limitations, the client-hosted listen server model, is a good fit for Tank Tactics, as it allows for more accessible multiplayer gaming while still providing a reasonably secure and manageable network architecture. This model is particularly well-suited for smaller-scale multiplayer games like Tank Tactics, where the number of players per match is limited, and the costs of running dedicated servers may outweigh the benefits.

\section{Matchmaking and Lobby Systems}

Text Here

\section{Unity Game Engine and Multiplayer Technologies}

% A possible section of your chapter
\section{Critique of the review} % Use this section title or choose a better one


% Pleae use this section
\section{Summary} 

