\chapter{Conclusions and Future Work}
\label{ch:con}
\section{Conclusions}
The Tank Tactics project successfully addressed the challenges associated with traditional online multiplayer games by leveraging Unity's Netcode for GameObjects (NGO) framework and Unity Gaming Services (UGS) Relay \& Lobby. The implementation of these technologies eliminated the need for manual port forwarding and IP address sharing, providing players with a seamless and accessible multiplayer experience.
\\
\noindent
\\
The project met its primary objective of developing an engaging and dynamic online multiplayer tank battle game for up to 20 players simultaneously. The core gameplay mechanics, including intuitive tank controls, strategic coin collection, and intense combat, were carefully designed to create an immersive and rewarding experience for players.
\\
\noindent
\\
The incorporation of advanced features, such as the real-time leaderboard, mini-map, healing zones, and bounty coins, added depth and variety to the gameplay, encouraging players to strategize and compete. The game's design drew inspiration from popular .io-style games and classic tank-based titles, providing a familiar and intuitive experience for players.
\\
\noindent
\\
The modular and well-structured approach to development ensured the maintainability and extensibility of the codebase. The separation of concerns through categorized scripts and prefabs facilitated easy debugging and future expansion of the game's features.
\\
\noindent
\\
Overall, Tank Tactics successfully demonstrated the power and versatility of Unity's multiplayer technologies in creating an accessible and engaging online multiplayer gaming experience. By addressing the common challenges faced by traditional multiplayer games.

\section{Future Work}
While Tank Tactics has achieved success in creating an engaging and accessible online multiplayer tank battle game, there are numerous avenues for future development and improvement.
\\
\noindent
\\
One exciting direction could be the introduction of multiplayer co-op team battles, allowing players to form squads and strategically collaborate against opposing teams. This game mode could foster teamwork, communication, and coordination among allies, injecting a fresh layer of depth and social interaction into the gameplay experience.
\\
\noindent
\\
The implementation of private lobbies would cater to players seeking more controlled and exclusive multiplayer sessions. This feature could enable friends, clans, or competitive groups to create private matches, fostering a tight-knit community and facilitating organized tournaments or events within the Tank Tactics ecosystem.
\\
\noindent
\\
Expanding the game's customization options through a tank customization system could prove highly appealing to players. The ability to personalize the appearance, colour schemes, and potentially even performance attributes of their tanks could heighten the sense of ownership and individuality, allowing players to express their unique styles while engaging in battles.
\\
\noindent
\\
Furthermore, the introduction of custom power-ups could dramatically alter the strategic landscape of Tank Tactics. Imagine players acquiring temporary abilities such as enhanced speed boosts, defensive shields, or even special weaponry. This mechanic could inject an element of unpredictability and dynamism into matches, forcing players to adapt their tactics on the fly and adding another layer of depth to the gameplay experience.
\\
\noindent
\\
The potential integration of dedicated server infrastructure could address scalability and performance concerns as the game's popularity grows. Dedicated servers would provide a robust architecture capable of supporting larger concurrent player counts while maintaining a smooth and consistent gameplay experience.
\\
\noindent
\\
Finally, performance optimization through code profiling and refinements should remain an ongoing priority, ensuring a consistently smooth and responsive experience as new features are introduced and the player population expands.